
\usepackage{ragged2e} %\justifying
\usepackage{paracol}
\usepackage[utf8]{inputenc} %Para acentos en UTF8 (Prueba: á é í ó ú Á É Í Ó Ú ñ Ñ)
\usepackage{url}
%\usepackage{mathtools}
\usepackage{graphicx}
\usepackage{caption}
\usepackage{float} % para que los gr\'aficos se queden en su lugar con [H]
\usepackage[fleqn]{mathtools} % \coloneqq, flalign
\usepackage{subcaption}
\usepackage{wrapfig}
\usepackage{soul,color} %\st{Hellow world}
\usepackage[fleqn]{amsmath} %para escribir funci\'on partida
\usepackage{blkarray}
\usepackage{hyperref} % para inlcuir links dentro del texto
\usepackage{tabu} 
\usepackage{comment}
\usepackage{amsfonts} % mathbb{N} -> conjunto de los n\'umeros naturales  
\usepackage{enumerate}
\usepackage{listings}
\usepackage[shortlabels]{enumitem} %  shortlabels option to have compatibility with the enumerate-like scheme for label
\usepackage{framed}
\usepackage{mdframed}
\usepackage{multicol}
\usepackage{transparent} % \transparent{1.0}
\usepackage{bm} 
\usepackage[makeroom]{cancel} % \cancel{} \bcancel{} etc
\usepackage[absolute,overlay]{textpos} %no funciona
\setlength{\TPHorizModule}{1mm} %128mm  mitad: 64 
\setlength{\TPVertModule}{1mm}	%96mm  mitad 48

\newcommand\hfrac[2]{\genfrac{}{}{0pt}{}{#1}{#2}} %\frac{}{} sin la linea del medio

\newtheorem{midef}{Definition}
\newtheorem{miteo}{Theorem}
\newtheorem{mipropo}{Proposition}

\usefonttheme[onlymath]{serif}


\usepackage{tikz} % Para graficar, por ejemplo bayes networks
\usetikzlibrary{bayesnet} % Para que ande se necesita copiar el archivo  tikzlibrarybayesnet.code.tex en la misma carpeta

%%%%%%%%%%%%%%%%%%%%%%%%%%%%%%%%%5
%
% Incompatibles con textpos
%
%\usepackage{todonotes}
%\usepackage{tikz} % Para graficar, por ejemplo bayes networks
%
%%%%%%%%%%%%%%%%%%%%%%%%%%%%%%%%%%



\usepackage[absolute,overlay]{textpos} %no funciona
\setlength{\TPHorizModule}{1mm} %128mm  mitad: 64 
\setlength{\TPVertModule}{1mm}	%96mm  mitad 48
% 
% 
\captionsetup[figure]{labelformat=empty}

\hypersetup{
    colorlinks=true,       % false: boxed links; true: colored links
    linkcolor=white,          % color of internal links (change box color with linkbordercolor)
    citecolor=green,        % color of links to bibliography
    filecolor=magenta,      % color of file links
    urlcolor=cyan           % color of external links
}
% 
% http://latexcolor.com/
\definecolor{lightseagreen}{rgb}{0.13, 0.7, 0.6.5}
\definecolor{greenblue}{rgb}{0.1, 0.55, 0.5}
\definecolor{redgreen}{rgb}{0.6, 0.4, 0.}
\definecolor{greenred}{rgb}{0.4, 0.7, 0.}
\definecolor{redblue}{rgb}{0.4, 0., .4}
\definecolor{tangelo}{rgb}{0.98, 0.3, 0.0}
\definecolor{git}{rgb}{0.94, 0.309, 0.2}
% 
\setbeamercolor{structure}{fg=greenblue}





\newcommand\Wider[2][3em]{%
\makebox[\linewidth][c]{%
  \begin{minipage}{\dimexpr\textwidth+#1\relax}
  \raggedright#2
  \end{minipage}%
  }%
}

\newenvironment{ejercicio}[1]{
% \setbeamercolor{block title}{bg=tangelo, fg=white}
\begin{exampleblock}{#1}
}{
\end{exampleblock}
}

\newenvironment{resumen}[1]{
\setbeamercolor{block title}{bg=git, fg=white}
\begin{block}{#1}
}{
\end{block}
}

\newenvironment{comando}{
\setbeamercolor{block body}{bg=git, fg=white}
\begin{block}{}
\begin{center}
\LARGE
\begin{texttt}
}{
\end{texttt}
\end{center}
\end{block}
}


